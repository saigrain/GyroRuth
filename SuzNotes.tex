\documentclass[11pt,a4paper]{article}
\usepackage{txfonts}
\usepackage{graphicx}
\usepackage[top=2cm,left=2cm,right=2cm,bottom=1.5cm,headsep=7mm,footskip=7mm]{geometry}
\usepackage[usenames,dvipsnames]{color}
\usepackage{natbib}
\parindent 0in
\parskip 1mm
\parsep 1mm
\newenvironment{packed_enum}{
\begin{enumerate}\vspace{-3mm}
\setlength{\itemsep}{0pt}
\setlength{\parskip}{0pt}
\setlength{\parsep}{0pt}
}{\end{enumerate}}\vspace{-4mm}
\newenvironment{packed_item}{
\begin{itemize}
\setlength{\itemsep}{1pt}
\setlength{\parskip}{0pt}
\setlength{\parsep}{0pt}
}{\end{itemize}}
\newenvironment{packed_desc}{
\begin{description}
\setlength{\itemsep}{1pt}
\setlength{\parskip}{0pt}
\setlength{\parsep}{0pt}
}{\end{description}}

\newcommand{\red}[1]{\textcolor{red}{#1}}
\begin{document}

\title{Notes on modelling gyrochronology model for Ruth's Kepler asteroseismic sample}
\author{Suzanne Aigrain}
\date{\today}

\maketitle

\section{Preliminaries}

\subsection{The data}

The dataset consists of $N$ stars, labelled by the index $n =
1,\ldots,N$. For each star, we have the following main observables:
\begin{packed_item}
\item the effective temperature $T$, derived from the star's spectrum,
  in degrees Kelvin;
\item the rotation period $P$, derived from the star's light curve, in
  days;
\item the age $A$, derived from asteroseismology, in Gyr.
\item the surface gravity $\log g$ \red{(not sure if it's derived from
    the star's spectrum, asteroseismology, or both)}, in
  g\,cm$^{3}$. For convenience of notation, we define $G=\log g$;
\end{packed_item}
All the above have measurement uncertainties which are assumed to be
known, Gaussian and independent, i.e.
\begin{equation}
p(\hat{X}_n|X_n) = \mathcal{N}(\hat{X}_n|X_n,\sigma_{X,n}^2)
\end{equation}
where $X$ represents any of the above variables, $X_n$ is the true
value of this variable in observation $n$, $\hat{X}_n$ is the noisy
observed value and $\sigma_{X,n}$ the associated measurement
uncertainty.

\subsection{The astrophysics}
\label{sec:astro}

The range of the key observables covered by our sample is as follows:
$5500 < T < 6500$, $1 <P < 50$, $1 < A < 10$, and $3.5 < G <
4.5$. Main sequence stars cooler (or bluer) than the so-called Kraft
break (which is somewhere in the range $6000 < T_K < 6500$), are
assumed to follow a unique gyrochronology relation relating $T$ (or
mass, or $B-V$ or $J-K$ colour), $P$ (or $v \sin i$) and $A$. The goal
of this study is to use the dataset to improve the calibration of this
gyrochronology relation.  A commonly adopted form for the
gyrochronology relation in the recent literature (see e.g.\
\citealt{bar07,mam+08}) is:
\begin{equation}
P = a \left[(B-V) - c\right]^b \, \times \, A^d,
\label{eq:gyro1}
\end{equation}
with typical values for the parameters $a \sim 0.4$, $b \sim
0.3$--0.6, $c \sim 0.4$--0.5 and $d \sim 0.5$--0.6. In this
expression, $c$ corresponds to the Kraft break, and the
gyronochronology relation is not defined for stars bluer than
this. Since we are working with $T$ rather than colour, we might adopt
a similar relation:
\begin{equation}
P = a (T - T_K)^b  \, \times \, t^d.
\label{eq:gyro2}
\end{equation}

Our sample also includes main sequence stars hotter than the Kraft
break, which show very little rotational evolution: they tend to have
relatively short rotation periods (1--10\,d) but aside from that their
periods display little dependence on either effective temperature or
age. Although the Kraft break is related to the size of the outer
convection zone, the effective temperature (or colour) at which it
occurs is not well established, and it is even possible that stars
close to the Kraft break could display either behaviour (rotational
evolution following the gyrochronology relation, or no evolution),
seemingly at random. Finally, our sample also contains stars which
have recently left the main sequence, i.e.\ sub-giants, which on the
contrary experience very rapid spin-down. Contrarily to the Kraft
break, we can assume that we know (from stellar evolution models) the
location of the main sequence turnoff as a function of effective
temperature and surface gravity.

\subsection{The statistical approach}

We want model the data in such a way as to 
\begin{packed_item}
\item enable us to compute a predictive distibution for the age of a
  star given estimates of its effective temperature and rotation
  period.
\item take into account the uncertainties on all three observables in
  our sample;
\item allow for the fact that there are three distinct populations in
  our sample, and that it may not be possible to uniquely assign a
  given star to a specific population;
\item readily enable us to incorporate other datasets (e.g.\ from
  clusters, including cooler stars) in the future.
\end{packed_item}

The first point above suggests that we should treat $A$ as the
dependent variable and $T$ and $P$ as the independent variables.  This
implies re-arranging Eq.~\ref{eq:gyro2}:
\begin{equation}
\log A = \alpha+ \beta \log \left(T-T_K \right) + \delta \log P,
\label{eq:gyro3}
\end{equation}
where $\alpha = - \log(a)/d$, $\beta = \log(b)/ d$, and $\delta=-1/d$.

The second point can be addressed by using a sampling approach to
marginalise over the uncertainties on the independent variables, as
described in Section~\ref{sec:noisy_plane}.  The third point suggests
either a composite model function, or a mixture model approach, as
described in Section~\ref{sec:mixture}. The fourth point is trivial in
a hierarchical Bayesian framework, but it does mean that we want our
model to be defined over a range of temperatures wider than the one in
our sample.

\section{Constructing the model}

We start by writing down by writing the likelihood marginalised over
the observational errors:
\begin{equation}
\label{eq:fullL}
  p(\{\hat{P}_n,\hat{A}_n,\hat{T}_n,\hat{G}_n\}|\theta,T_K) = 
  \prod_{n=1}^{N} \int p(\hat{A}_n,\hat{T}_n,\hat{P}_n,\log {g}_n,A_n,T_n,P_n,G_n|\theta,T_K) 
  {\rm d}A_n {\rm d}T_n {\rm d}P_n {\rm d}G_n
\end{equation}
where $\theta = \{\alpha,\beta\,\delta\}$ represents the parameters of
the gyrochronology relation.  The joint probability distribution is
then given by:
\begin{equation}
  \label{eq:jointprob}
  p(\hat{A}_n,\hat{T}_n,\hat{P}_n,\log {g}_n,A_n,T_n,P_n,G_n|\theta,T_K) = 
  p(A_n,T_n,P_n,G_n|\theta,T_K) p(\hat{A}_n|A_n) 
  p(\hat{T}_n|T_n) p(\hat{P}_n|P_n) p(\hat{G}_n|G_n),
\end{equation}
If we chose to treat the transition between the different populations
of stars in our sample as a step function, we can write the
marginalised likelihood for a single star as the sum of three
integrals corresponding to the three different regimes:
\begin{equation}
  \label{eq:L1}
  p(\hat{P}_n,\hat{A}_n,\hat{T}_n,\hat{G}_n|\theta,T_K)  = 
  \sum_{k=1}^3 \int_k p_k(A_n,T_n,P_n,G_n|\theta,T_K) 
  p(\hat{A}_n|A_n) p(\hat{T}_n|T_n) p(\hat{P}_n|P_n) p(\hat{G}_n|G_n),
  {\rm d}A_n {\rm d}T_n {\rm d}P_n {\rm d}G_n
\end{equation}
where $k=1$ corresponds to main sequence stars cooler than the Kraft
break, $k=2$ to main sequence stars hotter than the Kraft break, and
$k=3$ to stars which have left the main sequence, and $\int_k f(x)
{\rm d}x$ is the integral of $f(x)$ with respect to $x$, evaluated
over regime $k$. Note that each star may still have a finite
probability of belonging to any one of the individual regimes, because
of the uncertainties on the observables.

The joint probability distribution for the true observables takes a
different from in each of the three regimes:
\begin{eqnarray}
p_1(A_n,T_n,P_n,G_n|\theta,T_K) & = & p_1(T_n|T_K) p_1(G_n) p_1(P_n) p_1(A_n|T_n,P_n,\theta,T_K), \label{eq:composite1} \\
p_2(A_n,T_n,P_n,G_n|\phi_2,T_K) & = & p_2(T_n|T_K) p_2(G_n) p_2(P_n|phi_2) p_2(A_n),~{\rm and}\label{eq:composite2} \\
p_3(A_n,T_n,P_n,G_n|\phi_3) & = & p_3(T_n) p_3(G_n) p_3(P_n|\phi_3) p_3(A_n)\label{eq:composite3},
\end{eqnarray}
where:
\begin{equation}
p_1(A_n|T_n,P_n,\theta,T_K) = \delta \left\{ \log A_n - \left[ \alpha+ \beta \log \left(T-T_K \right) + \delta \log P_n \right] \right\}.
\end{equation}
These equations merit a few comments:
\begin{packed_item}
\item The form of Eqs.\,(\ref{eq:composite2}) and
  (\ref{eq:composite3}) implies that there is no correlation between
  $P$ and $T$ or $A$ in regimes 2 and 3. This is reasonable for regime
  2, where we do not expect any rotational evolution, as discussed in
  Section~\ref{sec:astro}.  It is less appropriate, at first sight,
  for regime 3, where on the contrary we expect very rapid rotational
  evolution, but this regime is not the focus of the present study, we
  are less interested in modelling it in detail.
\item The priors are defined separately over each regime, for two
  reasons. First, the priors over $T$ and $G$ need to reflect the
  different ranges for these variables in the three
  regimes. Additionally, we can choose different period priors in the
  different regimes. For example we might adopt a log-normal prior
  with a different mean and/or variance in each regime, or a
  log-uniform prior defined over a different range, to reflect our
  expectation that stars tend to have longer or shorter rotation
  periods in a given regime. In Eqs.~(\ref{eq:composite2}) and
  (\ref{eq:composite3}), we have specified a prior with
  hyper-parameters $\phi$, which could be inferred or marginalised
  over at the same time as $\theta$ and $T_K$.
\item The use of mutually independent priors for $T$, $G$ and $A$ in
  Eqs.\,(\ref{eq:composite1}--\ref{eq:composite3}) also implies that
  there is no correlation between these quantities. Of course, this
  isn't true: there are built-in correlations because of stellar
  evolution. In principle, it would be more adequate to construct a
  joint, age dependent prior on $T$ and $G$, using theoretical
  evolutionary models, assuming an initial mass distribution and a
  star formation history (i.e. a prior on $A$). However, that would
  make the results dependent on the very models we are trying to test,
  and in any case the quality of our date probably doesn't warrant
  such a level of sophistication. So for now I would say use a uniform
  prior on $T$ and $G$ and a log-uniform prior on $A$.
\end{packed_item}

The three terms in Eq.\,(\ref{eq:L1}) become:
\begin{eqnarray}
p_1(\hat{P}_n,\hat{A}_n,\hat{T}_n,\hat{G}_n|\theta,T_K) & = & \int_1 p_1(T_n|T_K) p_1(G_n) p_1(P_n) p_1(A_n|T_n,P_n,\theta,T_K)
p(\hat{A}_n|A) p(\hat{T}_n|T_n) p(\hat{P}_n|P_n) p(\hat{G}_n|G_n) {\rm d}A_n {\rm d}T_n {\rm d}P_n {\rm d}G_n
\label{eq:compo1} \\
p_2(\hat{P}_n,\hat{A}_n,\hat{T}_n,\hat{G}_n|\phi,T_K) & = & \int_2 p_2(T_n|T_K) p_2(G_n) p_2(P_n|\phi) p_2(A_n) 
p(\hat{A}_n|A) p(\hat{T}_n|T_n) p(\hat{P}_n|P_n) p(\hat{G}_n|G_n) {\rm d}A_n {\rm d}T_n {\rm d}P_n {\rm d}G_n
\label{eq:compo2} \\
p_3(\hat{P}_n,\hat{A}_n,\hat{T}_n,\hat{G}_n) & = & \int_3 p_3(T_n) p_3(G_n) p_3(P_n) p_3(A_n) 
p(\hat{A}_n|A) p(\hat{T}_n|T_n) p(\hat{P}_n|P_n) p(\hat{G}_n|G_n) {\rm d}A_n {\rm d}T_n {\rm d}P_n {\rm d}G_n
 \label{eq:compo3} 
\end{eqnarray}
In the single regime case, if using fixed priors, these act as
constant normalisation constants, and can be ignored. However, if we
have multiple regimes with different priors, and / or if the priors
depend on the (hyper-)parameter we wish to infer or marginalise over,
as is the case here, we must keep track of them in the calculation.

To evaluate the marginalised likelihood, we are going to use
importance sampling. Say we have $J_n$ samples:
\begin{eqnarray}
T_n^{(j)} & \sim & p(T_n|\hat{T}_n) \nonumber \\
P_n^{(j)} & \sim & p(P_n|\hat{P}_n)  \nonumber  \\
A_n^{(j)} & \sim & p(T_n|\hat{A}_n)  \nonumber  \\
G_n^{(j)} & \sim & p(G_n|\hat{G}_n) 
\end{eqnarray}
Eq.\,(\ref{eq:compo1}) can be approximated as
\begin{equation}
p_1(\hat{P}_n,\hat{A}_n,\hat{T}_n,\hat{G}_n|\theta,T_K) \approx \frac{1}{J_n}\sum_{j=1}^N f_1(T_n^{(j)},G_n^{(j)}|T_K) p_1(T^{(j)}_n|T_K) p_1(G^{(j)}_n) p_1(P^{(j)}_n) p(\hat{A}_n|\tilde{A}_n^{(j)})
\end{equation}
where $f_1(T_n^{(j)},G_n^{(j)}|T_K) = 1$ if the sample falls in regime
1 and 0 otherwise, and $\tilde{A}$ is the value of $A$ predicted by
the gyrochronology relation given $T^{(j)}$ and $P^{(j)}$:
\begin{equation}
\log \tilde{A}^{(j)}_n = \alpha+ \beta \log \left(T^{(j)}_n-T_K \right) + \delta \log P^{(j)}_n.
\end{equation}
Similarly, Eq.\,(\ref{eq:compo2}) can be approximated as
\begin{equation}
p_2(\hat{P}_n,\hat{A}_n,\hat{T}_n,\hat{G}_n|\phi,T_K) \approx \frac{1}{J_n}\sum_{j=1}^N f_2(T_n^{(j)},G_n^{(j)}|T_K) p_2(T^{(j)}_n|T_K) p_2(G^{(j)}_n) p_2(P^{(j)}_n|\phi) p_2(A^{(j)}_n).
\end{equation}
where $f_2(T_n^{(j)},G_n^{(j)}|T_K) = 1$ if the sample falls in regime
2 and 0 otherwise.  Finally, Eq.\,(\ref{eq:compo3}) can be
approximated as
\begin{equation}
p_3(\hat{P}_n,\hat{A}_n,\hat{T}_n,\hat{G}_n) \approx \frac{1}{J_n}\sum_{j=1}^N f_3(T_n^{(j)},G_n^{(j)}) p_3(T^{(j)}_n) p_3(G^{(j)}_n) p_3(P^{(j)}_n) p_3(A^{(j)}_n).
\end{equation}
where $f_3(T_n^{(j)},G_n^{(j)}) = 1$ if the sample falls in regime 3
and 0 otherwise.

The full marginalised likelihood is then
\begin{equation}
\log p(\{\hat{A}_n,\hat{T}_n,\hat{P}_n,\hat{G}_n\}|\theta,\phi,m)
\approx \\sum_{i=1}^n \log
\left(\frac{1}{J_n}\sum_{j=1}^N \mathcal{P}^{(j)} \right)
\end{equation}
where 
\begin{eqnarray}
\mathcal{P}^{(j)} & = &  f_1(T_n^{(j)},G_n^{(j)}|T_K) p_1(T^{(j)}_n|T_K) p_1(G^{(j)}_n) p_1(P^{(j)}_n) p(\hat{A}_n|\tilde{A}_n^{(j)}) + \nonumber \\
&& f_2(T_n^{(j)},G_n^{(j)}|T_K) p_2(T^{(j)}_n|T_K) p_2(G^{(j)}_n)
p_2(P^{(j)}_n|\phi) p_2(A^{(j)}_n) + \nonumber \\
&& f_3(T_n^{(j)},G_n^{(j)}) p_3(T^{(j)}_n) p_3(G^{(j)}_n) p_3(P^{(j)}_n) p_3(A^{(j)}_n) 
\end{eqnarray}
\end{document}